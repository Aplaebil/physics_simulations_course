\chapter{Introduction}

\section{Why are Simulations Used?}
Text

\section{Python}
Text
\section{A Bit About Git}
Text

\section{Some Mathematical Background}
Text

\section{Harmonic Oscillator}
\subsection{Simple Harmonic Oscillator}
Many systems in physics present a simple, periodic (repeating) motion. One such system is a simple mass-less spring connected to a mass $m$ and allowed to move in a single dimension only. If we ignore the effects of gravity, the only force acting on the mass arises from the spring itself: the more we pull or push the spring, the stronger it will resist to that change. This resistant force is given by
\begin{equation}
  F = -kx,
  \label{eq:spring_force}
\end{equation}
where $k$ is the \textbf{spring constant}, and $x$ is the amount by which the spring contracts or expands relative to its rest position $x_{0}$. In \autoref{fig:simple_spring} we show three cases: when the mass is at the springs rest position, $x_{m}=x_{0}$, the spring applies no force on it. When the mass is displaced by a positive amount $\Delta x>0$, the spring applied a \textit{negative} force on it: $F=-k\Delta x<0$ (see note \autoref{note:negative_force}). And when the mass is displaced such that it contracts the spring, $\Delta x<0$ and thus the force applied by the spring is positive, $F=k\Delta x>0$.

\begin{note}{Negative and positive forces}{negative_force}
  Recall that in this context, negative force means a force in the negative $x$ direction.
\end{note}

\begin{figure}
  \begin{center}
    \begin{tikzpicture}
      \coordinate (x0) at (3,1.5);
      \draw[thick, dashed, black] (x0 |- 0,1) node [above] {$x_0$} -- ++(0,-8);
      \draw[thick, dashed,-stealth] (-0.5, 1.7) -- ++(3,0) node [midway, above] {positive $x$ direction};

      \draw[line width=5mm, black!30] (-0.25,1) -- ++(0,-1.75) -- ++(6.25,0);
      \draw[thick] (0,1) -- ++(0,-1.5) -- ++(6,0);
      \draw[thick, fill=xred!50] (3,0) circle (0.5) node (m1) {$m$};
      \draw[springcoil, xdarkblue] (0,0) -- ($(m1)-(0.5,0)$);

      \draw[line width=5mm, black!30] (-0.25,-2) -- ++(0,-1.75) -- ++(6.25,0);
      \draw[thick] (0,-2) -- ++(0,-1.5) -- ++(6,0);
      \draw[thick, fill=xred!50] (5,-3) circle (0.5) node (m2) {$m$};
      \draw[springcoil, xdarkblue] (0,-3) -- ($(m2)-(0.5,0)$);

      \draw[line width=5mm, black!30] (-0.25,-5) -- ++(0,-1.75) -- ++(6.25,0);
      \draw[thick] (0,-5) -- ++(0,-1.5) -- ++(6,0);
      \draw[thick, fill=xred!50] (1.5,-6) circle (0.5) node (m3) {$m$};
      \draw[springcoil, xdarkblue] (0,-6) -- ($(m3)-(0.5,0)$);

      \draw[xred, thick, cap=round, decorate, decoration={brace, amplitude=3pt, raise=3pt}] (m1 |- 0,-2.5) -- (m2 |- 0,-2.5) node[midway, above, yshift=5pt]{$\Delta x_{1}>0$};
      \draw[xred, thick, cap=round, decorate, decoration={brace, amplitude=3pt, raise=3pt, mirror}] (m1 |- 0,-5.5) -- (m3 |- 0,-5.5) node[midway, above, yshift=5pt]{$\Delta x_{2}<0$};

      \draw[-stealth, thick, xdarkblue] (2.5,-2) -- ++(-1,0) node[midway, above] {$F_{1}<0$};
      \draw[-stealth, thick, xdarkblue] (0.5,-5) -- ++(0.6,0) node[midway, above] {$F_{2}>0$};
    \end{tikzpicture}
  \end{center}
  \caption{A simple spring-mass system with spring constant $k$ and a mass $m$. The top figure shows the spring at rest - i.e. when the mass is located at position $x_{0}$ the spring applies no force on the mass (since $\Delta x = x_{m}-x_{0}=0$). The middle figure show the spring being at a \textit{positive} displacement $\Delta x_{1}>0$, causing the spring to pull back with a negative force $F_{1}=-k\Delta x_{1}$. The bottom picture shows the spring contracting by $\Delta x_{2}<0$, casing the spring to apply a positive force $F_{2}=-k\Delta x_{2}$ on the mass.}\label{fig:simple_spring}
\end{figure}

Using Newtons second law of motion (REF?) with $x$ being the displacement from the rest position $x_{0}$, we get the relation
\begin{equation}
  a = -kx,
  \label{eq:simple_spring_newton_2nd_law}
\end{equation}
but since $a = \od{v}{t} = \od[2]{x}{t}$, we can re-write \autoref{eq:simple_spring_newton_2nd_law} as
\begin{equation}
  \od[2]{x}{t} = -kx,
  \label{eq:simple_spring_newton_2nd_law_as_derivative}
\end{equation}
or even more succinctly as
\begin{equation}
  \ddot{x} = -kx.
  \label{eq:simple_spring_newton_2nd_law_as_ddot}
\end{equation}

\autoref{eq:simple_spring_newton_2nd_law_as_ddot} is one of the simplest possible 2nd-order \textit{ordinary} differential equations. Its solution is a combination of the two basic trigonometric equations:
\begin{equation}
  x(t) = c_{1}\sin\left(\alpha t\right) + c_{2}\cos\left(\beta t\right),
  \label{eq:harmonic_solution_ode}
\end{equation}
where $c_{1}$ and $c_{2}$ are constants which we can find using the \textit{starting conditions} (see note \autoref{note:ode_start_cond}), and $\alpha,\beta$ are parameters of the motion. Since function arguments must be unitless, these two parameters also cause the total quantity inside the trigonometric functions to be unitless by having units corresponding to \SIe{1\over time}. For example, if we measure the time in \SIe{\second}, then the units of $\alpha$ and $\beta$ are \SIe{\per\second} = \SIe{\hertz}.

\begin{note}{Starting conditions for solving ODEs}{ode_start_cond}
  Recall that in order to completely solve an ordinary differential equation of order $n$ we must have $n$ starting conditions.
\end{note}

In the case where the initial position of the mass is $x_{0}$ and the initial velocity is $\dot{x}_{0}=0$ we get the simplified solution
\begin{equation}
  x(t) = x_{0}\cos(\omega t),
  \label{eq:harmonic_normal_solution}
\end{equation}
where $\omega=\sqrt{\frac{k}{m}}$. The plot of the position $x(t)$ of the mass is given in \autoref{fig:simple_spring_plot}. Note how, when displayed side-by-side using the same time axes, the position plot is \enquote{lagging behind} the velocity plot by $\phi=\frac{\pi}{2}$: when we start the motion, the velocity is $0$ and then starts to increase in the negative direction (i.e. the mass is moving to the left). The position is at its maximum at that time, and decreases from $x_{\max}$ at $t=0$ to $x=0$ at $t=\frac{\pi}{2}$, while still staying positive. At $t=\frac{\pi}{2}$ the velocity reaches its maximum negative value, $\dot{x}=-v_{\max}$ (which is $1$ in units of $\frac{1}{v_{\max}}$), and the position becomes negative (since it is to the left of the rest position of the spring).

\begin{figure}
  \centering
  \begin{tikzpicture}
    \begin{axis}[
      name=pos_vs_time,
      graph2d,
      width=10cm, height=4cm,
      xmin=0, xmax=6*pi+0.5,
      ymin=-1.1, ymax=1.1,
      domain={0:6*pi+0.5},
      xlabel=$\frac{t}{2\pi\omega}$,
      ylabel=$\frac{x}{x_{0}}$,
      xtick={2*pi,4*pi,6*pi},
      ytick={-1,0,1},
      xticklabels={$2\pi$,$4\pi$,$6\pi$}
    ]
    \addplot[function={xblue}] {cos(x)};
    \end{axis}
    \begin{axis}[
      at=(pos_vs_time.below south west),
      anchor=north west,
      yshift=-1cm,
      graph2d,
      width=10cm, height=4cm,
      xmin=0, xmax=6*pi+0.5,
      ymin=-1.1, ymax=1.1,
      domain={0:6*pi+0.5},
      xlabel=$\frac{t}{2\pi\omega}$,
      ylabel=$\frac{\dot{x}}{\dot{x}_{\max}}$,
      xtick={2*pi,4*pi,6*pi},
      ytick={-1,0,1},
      xticklabels={$2\pi$,$4\pi$,$6\pi$}
    ]
    \addplot[function={xred}] {-sin(x)};
  \end{axis}
  \end{tikzpicture}
  \caption{Simple harmonic oscillator, top (blue): position $x$ vs. time $t$. The axes units are such that a full period of the oscillation takes $\Delta t=2\pi$, and that the minimum and maximum values of the position are $\pm1$, respectively. Bottom (red): velocity vs. time on the same time axes, and a velocity axis which is scaled such that $v_{\max}=1$.}
  \label{fig:simple_spring_plot}
\end{figure}

It would be useful to understand how does the spring-mass system evolve given a specific combination of position and velocity. This can be done by plotting the \textbf{phase space} of the system (\autoref{fig:harmonic_oscillator_phase_space}): on the horizontal axis we specify the position $x$ of the mass, and on the vertical axis the velocity $\dot{x}$.

\begin{figure}
  \centering
  \begin{tikzpicture}
    \begin{axis}[
      graph2d,
      width=10cm, height=10cm,
      xmin=-3, xmax=3,
      ymin=-3, ymax=3,
      axis equal=true,
      domain={-3:3},
      xlabel=$\frac{x}{X_{0}}$,
      ylabel=$\frac{\dot{x}}{\dot{X}_{\max}}$,
      xtick={-3,-2,...,3},
      ytick={-3,-2,...,3},
      samples=100,
    ]
    \draw[very thick, xpurple] (0,0) circle (0.5);
    \draw[very thick, xblue] (0,0) circle (1);
    \draw[very thick, xgreen] (0,0) circle (1.5);
    \draw[very thick, xorange] (0,0) circle (2);
    \draw[very thick, xred] (0,0) circle (2.5);
    \end{axis}
  \end{tikzpicture}
  \caption{Phase space plot of simple harmonic oscillators with different momenta (either their masses are different, or their initial distance are different). The axes are scaled such that their units are the initial position $X_{0}$ and maximum velocity $\dot{X}_{0}$, respectively, of the second oscillator (trajectory drawn in blue).}
  \label{fig:harmonic_oscillator_phase_space}
\end{figure}

In the case of a simple harmonic oscialltor, the evolution of the system is shown on the phase space plot as ellipses - and when we use normalized coordinates ($\tilde{x}=\frac{x}{x_{0}}$ and $\tilde{x}=\frac{\dot{x}}{\dot{X}_{\max}}$), the ellipses turn into perfect circles - just like we see in \autoref{fig:harmonic_oscillator_phase_space}. These circles or ellipses are \underline{paths of constant energy}: this can be seen when we \enquote{translate} the plot to have the potential energy and kinetric energy as horizontal and vertical axes, respectively.

Recall that for close mechanical systems, the total energy $E$ is given by the sum of the sum of the \textit{kinetic energy} $K$ and the \textit{potential energy} $U$:
\begin{equation}
  E = K + U.
  \label{eq:total_energy}
\end{equation}
The kinetic energy is a function of the velocity $v=\dot{x}$:
\begin{equation}
  K = \frac{1}{2}\dot{x}^{2} = \frac{p^{2}}{2m},
  \label{eq:kinetric_energy}
\end{equation}
and the potential energy is a function of the position $x$ via the force $F$:
\begin{equation}
  F = -\od{U}{x}.
  \label{eq:force_from_potential}
\end{equation}
In the case of an harmonic oscillator $F=-kx$, and therefore
\begin{equation}
  U = \int\displaystyle F \dif x = -k\int\displaystyle x \dif x = -\frac{1}{2}kx^{2} + c.
  \label{eq:potential_energy_harmonic_oscillator}
\end{equation}
Since we can add to the potential any constant with no change to the force derived from it (remember that the derivative of a constant is zero), we can simply set the potential at $x=0$ to $U\left(x=0\right)=0$, meaning that the integral constant is $c=0$. Altogether, we get the following system energy:
\begin{equation}
  E = \frac{p}{2m} - \frac{1}{2}kx^{2},
  \label{eq:total_energy_harmonic_oscillator}
\end{equation}
Essentially, this is equal to applying the following transformations to the phase space plot:
\begin{align}
  x &\to -\frac{1}{2}kx^{2},\\
  \dot{x} &\to \frac{p^{2}}{2m} = \frac{1}{2m}{m^{2}\dot{x}^{2}} = \frac{1}{2}m\dot{x}^{2}.
\end{align}
Since the transformation \enquote{streches} both axes by the same power (up to the constants $m$ and $k$), the shapes remain the same as in the original plot.

The total energy of the system can then be extracted from the radius of each circle path:
\begin{equation}
  E = \sqrt{K^{2}+U^{2}} = \sqrt{\frac{p^{2}}{4m^{2}} + \frac{k^{2}x^{4}}{4}} = \frac{1}{2}\left(\frac{p^{2}}{m^{2}}+k^{2}x^{4}\right).
  \label{eq:label}
\end{equation}
When $K=0$ (i.e. $p=0$) the entire energy is stored as potential energy:
\begin{equation}
  E = \frac{1}{2}\sqrt{k^{2}x^{4}} = \frac{1}{2}kx^{2}.
  \label{eq:label}
\end{equation}
And when $U=0$ (i.e. $x=0$) the entire energy is stored as kinetic energy:
\begin{equation}
  E = \frac{1}{2}\sqrt{\frac{p^{2}}{m^{2}}} = \frac{p}{2m} = \frac{1}{2}m\dot{x}^{2}.
  \label{eq:label}
\end{equation}

\subsection{Damped Harmonic Oscillator}
We can make the harmonic oscillator model a bit more realistic if we add a damping force proportional to the velocity of the mass. This corresponds e.g. to introducing friction into the model. The form of the damping force is $F_{\text{damp}}(t)=-cv(t)=-c\dot{x}$, for some real damping coefficient $c\geq0$ (in the case where $c=0$ we get back the simple harmonic oscillator model). Thus, the overall Newton 2nd law equation of the system has the following form:
\begin{equation}
  m\ddot{x} = -c\dot{x}-kx.
  \label{eq:damped_harmonic_oscillator}
\end{equation}

The term $-c\dot{x}$ always acts in the opposite direction to the velocity due to the minus sign. Re-arranging \autoref{eq:damped_harmonic_oscillator} we get the more \enquote{canonical} form
\begin{equation}
  m\ddot{x} + c\dot{x} + kx = 0,
  \label{eq:damped_harmonic_oscillator_different_form}
\end{equation}
We then use the solution $x(t)=\Eu{\lambda t}$, subtituting it into \autoref{eq:damped_harmonic_oscillator_different_form}:
\begin{align}
  F_{\text{total}}(t) &= \lambda^{2}m\Eu{\lambda t} + c\Eu{\lambda t} + k\Eu{\lambda t}\\
  &= \Eu{\lambda t}\left(m\lambda^{2}+c\lambda+k\right)\\
  &= 0.
  \label{eq:2nd_order_ODE_polynomial}
\end{align}

Since \autoref{eq:2nd_order_ODE_polynomial} is true for all $t$ (and in any case $\Eu{\lambda t}\neq 0$ for all $t$), for the equation to be true it must be that the quadratic eqaution in $\lambda$ equals zero. Using the quadratic formula we get that
\begin{equation}
  \lambda_{1,2} = \frac{-c\pm\sqrt{c^{2}-4km}}{2m}.
  \label{eq:2nd_order_ODE_polynomial_solution}
\end{equation}
For the solutions to $\lambda$ to be real numbers, $c^{2}\geq4km$ - otherwise the term in the square root is negative. When $c^{2}>4km$ we call the system \textbf{overdamped}, and the case where $c^{2}=4km$ is the \textbf{critical damping} value. MORE TEXT.

The overall position vs. time relationship of the underdamped system is given by
\begin{equation}
  x(t) = x_{0}\Eu{-\frac{c}{2m}t}\cos\left(\omega t\right),
  \label{eq:underdampled_oscillation}
\end{equation}
where as before $\omega=\sqrt{\frac{k}{m}}$ (see \autoref{fig:underdamped_oscillation}).

\begin{figure}
  \centering
  \begin{tikzpicture}
    \begin{axis}[
      name=pos_vs_time,
      graph2d,
      width=10cm, height=4cm,
      xmin=0, xmax=6*pi+0.5,
      ymin=-1.25, ymax=1.25,
      domain={0:6*pi+0.5},
      xlabel=$\frac{t}{2\pi\omega}$,
      ylabel=$\frac{x}{x_{0}}\mid\frac{\dot{x}}{\dot{x}_{0}}$,
      xtick={2*pi,4*pi,6*pi},
      ytick={-1,...,1},
      xticklabels={$2\pi$,$4\pi$,$6\pi$}
    ]
    \addplot[function={xblue}] {exp(-0.25*x)*cos(x)};
    \addplot[function={xred}] {-exp(-0.25*x)*(sin(x)+cos(x))};
    \end{axis}
  \end{tikzpicture}
  \caption{Underdamped harmonic oscillation\ldots}
  \label{fig:underdamped_oscillation}
\end{figure}

The phase space plot of the damped system shows\ldots

\begin{figure}
  \centering
  \begin{tikzpicture}
    \begin{axis}[
      graph2d,
      width=10cm, height=10cm,
      xmin=-1.1, xmax=1.1,
      ymin=-1.1, ymax=1.1,
      axis equal=true,
      xlabel=$x$,
      ylabel=$\dot{x}$,
      xtick={-1,...,1},
      ytick={-1,...,1},
      samples=250,
    ]
    \addplot[domain=0:10*pi, function={xpurple}, variable=\t] ({exp(-0.1*t)*cos(t)},{exp(-0.1*t)*sin(t)});
  \end{axis}
  \end{tikzpicture}
  \caption{Phase space plot of an underdamped harmonic oscillator\ldots}
  \label{fig:underdamped_harmonic_oscillator_phase_space}
\end{figure}


\subsection{Simulating Harmonic Oscillators in Python}
Text
