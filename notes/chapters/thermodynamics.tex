\chapter{Thermodynamics}

\section{Preface}
Text text text

\section{Ideal gas}
\subsection{Model: perfectly elastic spheres}
Text text text

Consider two solid spheres which have a single point of contact $\bm{A}$. Let $m_{1}, r_{1}, \vec{x}_{1}$ and $\vec{v}_{1}$ be the mass, radius, position and velocity of the first sphere, and $m_{2}, r_{2}, \vec{x}_{2}$ and $\vec{v}_{2}$ the respective quantities for the second sphere (\autoref{fig:elastic_collision}). The line connecting the centers of the two spheres is in the direction $\hat{n}$ (without loss of generality let us assume that the normal vector $\hat{n}$ points from $\vec{x}_{1}$ towards $\vec{x}_{2}$). The unit vector $\hat{t}$ is orthogonal to $\hat{n}$ (and without loss of generality we will assume that it is oriented counter-clockwise from $\hat{n}$).

\begin{figure}
	\begin{center}
		\begin{tikzpicture}
			\coordinate (xA) at (0, 0);
			\pgfmathsetmacro{\rA}{2}
			\pgfmathsetmacro{\rB}{1.5}
			\pgfmathsetmacro{\th}{45}
			\coordinate (xB) at ({(\rA+\rB)*cos(\th)},{(\rA+\rB)*sin(\th)});
			\coordinate (uAB) at ({\rA*cos(\th)},{\rA*sin(\th)});
			\draw[thick, dashed, black!50] (xA) -- (xB);

			\tikzset{
				sphere/.style={thick, fill=#1, fill opacity=0.3},
				radius/.style={thick, dashed, draw=#1},
			}
			\draw[sphere={xblue}]  (xA) circle (\rA);
			\draw[sphere={xgreen}] (xB) circle (\rB);
			\fill (xA) circle (0.05) node [above left]  {$m_{1}$};
			\fill (xB) circle (0.05) node [above right] {$m_{2}$};
			\draw[radius={xblue!75}] (xA) -- ++(0,-\rA) node [xblue!75, midway, right] {$r_{1}$};
			\draw[radius={xdarkgreen!75}] (xB) -- ++(0,-\rB) node [xdarkgreen!75, midway, right]  {$r_{2}$};

			\draw[vector={xblue}] (xA) -- ++(3,-0.5) node [pos=1.1] {$\vec{v}_{1}$};
			\draw[vector={xdarkgreen}] (xB) -- ++(-2,1.0) node [pos=1.1] {$\vec{v}_{2}$};

			\draw[vector={xred}] (uAB) -- ++({cos(\th)},{sin(\th)}) node [midway, above left] {$\hat{n}$};
			\draw[vector={xred}] (uAB) -- ++({cos(\th+90)},{sin(\th+90)}) node [pos=1.1] {$\hat{t}$};
		\end{tikzpicture}
	\end{center}
	\caption{Text text text}
	\label{fig:elastic_collision}
\end{figure}

Note that we did not define a coordinate system, nor the number of dimensions $d$ for the problem. The only restriction is that $d\geq2$.

Conservation of momentum means that the velocities of the spheres following the collision, $\vec{u}_{1},\vec{u}_2$, are related to their velocities before the collision by
\begin{equation}
	\begin{aligned}
		                 & m_{1}\vec{v}_{1} + m_{2}\vec{v}_{2} = m_{1}\vec{u}_{1} + m_{2}\vec{u}_{2}                   \\
		\Rightarrow\quad & m_{1}\left(\vec{v}_{1} - \vec{u}_{1}\right)  = m_{2}\left(\vec{u}_{2} - \vec{v}_{2}\right).
	\end{aligned}
	\label{eq:elastic_collision_coservation_of_momentum}
\end{equation}

Conservation of energy means that the velocities are also related by
\begin{equation}
	\begin{aligned}
		                 & \frac{1}{2}m_{1}\norm{\vec{v}_{1}}^{2} + \frac{1}{2}m_{2}\norm{\vec{v}_{2}}^{2} = \frac{1}{2}m_{1}\norm{\vec{u}_{1}}^{2} + \frac{1}{2}m_{2}\norm{\vec{u}_{2}}^{2} \\
		\Rightarrow\quad & m_{1}\left(\norm{\vec{v}_{1}}^{2} - \norm{\vec{u}_{1}}^{2}\right) = m_{2}\left(\norm{\vec{u}_{2}}^{2} - \norm{\vec{v}_{2}}^{2}\right).
	\end{aligned}
	\label{eq:elastic_collision_coservation_of_energy}
\end{equation}

However, the forces involved in the collision can not have a component in the $\hat{t}$ direction, and are limited to only point in the $\hat{n}$ direction. Therefore, we can reduce the problem to this direction only by projecting all velocities involved in the problem on $\hat{n}$, i.e. \autoref{eq:elastic_collision_coservation_of_momentum} becomes
\begin{equation}
	m_{1}\innerproduct{\vec{v}_{1}}{\hat{n}} + m_{2}\innerproduct{\vec{v}_{2}}{\hat{n}} = m_{1}\innerproduct{\vec{u}_{1}}{\hat{n}} + m_{2}\innerproduct{\vec{u}_{2}}{\hat{n}}.
	\label{eq:elastic_collision_coservation_of_momentum_projection}
\end{equation}

TEXT TEXT TEXT

\begin{equation}
	\begin{aligned}
		\vec{u}_{1} & = \vec{v}_{1} - \frac{2m_{2}}{m_{1}+m_{2}}\innerproduct{\vec{v}_{1}-\vec{v}_{2}}{\hat{n}}\hat{n}, \\
		\vec{u}_{2} & = \vec{v}_{2} + \frac{2m_{1}}{m_{1}+m_{2}}\innerproduct{\vec{v}_{1}-\vec{v}_{2}}{\hat{n}}\hat{n}.
	\end{aligned}
\end{equation}

To avoid reduntant calculations, we can factor out the common quantity of both velocities:
\begin{equation}
	K = \frac{2}{m_{1}+m_{2}}\innerproduct{\vec{v}_{1}-\vec{v}_{2}}{\hat{n}}\hat{n},
	\label{eq:elastic_collision_common_quantity}
\end{equation}
yielding
\begin{equation}
	\begin{aligned}
		\vec{u}_{1} & = \vec{v}_{1}-Km_{2}, \\
		\vec{u}_{2} & = \vec{v}_{2}+Km_{1}.
	\end{aligned}
	\label{eq:elastic_collision_final_equation}
\end{equation}
